%%% Research Diary - Entry
\documentclass[11pt,a4paper]{article}

% Working date: The date this entry is describing. Not necessary the date of last edit.
\newcommand{\workingDate}{\textsc{@year $|$ @MONTH $|$ @dday}}

% Name and institution must preceed call to researchdiary.sty package
\newcommand{\userName}{Pierre Bréchet}
\newcommand{\institution}{TUM}

% The guts and glory
\usepackage{research_diary}
\usepackage{usrcmd}
\bibliography{bibliography}

% Begin document.
% Use \logoPNG or \logoEPS. If compiling with PDFTeX, use \logoPNG
\begin{document}
% \logoPNG

{\Huge @MONTH @day}

\section{CelebA}

The CelebA dataset should be mastered. Here is the first outline for testing the dataset.

\subsection{Parameters}

Distance: L1 or cosine. The cosine seems to work better according to previous work but it is not a proper distance, so our theoretical analysis is somewhat improper.

Latent space: the InfoGAN paper uses a 10x10 categorical dimension and the rest noise (128 noise). The original CelebA dataset identified 40 categorical (binary) attributes to the images. Therefore, another possibility would be 40x2 categorical, a few continuous for the skin color, etc, and a few more for the noise (42).

Architecture for the newtork: the infoGAN has a relatively shallow network to train on CelebA as well, but uses 32x32 images (although display 64x64 images in their publication). So either use a 32x32 architecture as InfoGAN and afterwards enlarge the pictures, or train on 64x64 pictures with a different architecture (DCGAN).

For today entropic regularization is used, although quadratic might yield much more stable results.

Check from previous notes which batch-normalization should be in use, which bias, etc.

\begin{itemize}

    \item
        Batch normalization in all layers of D especially if the regularization is entropic
    \item
        No bias in D, bias in G
    \item
        Gradient mode set to ``{FE}'' in the update of the networks, and
        actually update the weights in the second pass for Sinkhorn (check with the pub directory)
\end{itemize}

\section{Correct the dataset}

Do not use the HDF5 file for loading the datasets. Instead, use the tutorial
explained \texttt{Dataset} class from \texttt{torch.utils}. The images are
stored in binary format and loaded on the fly (allow for multiprocessing of the
data). The previous argument (for having HDF5) was a speed issue, but it does
not seem to be the case actually.

\printbibliography{}
\end{document}
