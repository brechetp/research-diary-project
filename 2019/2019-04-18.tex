%%% Research Diary - Entry
\documentclass[11pt,a4paper]{article}

% Working date: The date this entry is describing. Not necessary the date of last edit.
\newcommand{\workingDate}{\textsc{2019 $|$ April $|$ 24}}

% Name and institution must preceed call to researchdiary.sty package
\newcommand{\userName}{Pierre Bréchet}
\newcommand{\institution}{TUM}

% The guts and glory
\usepackage{research_diary}
\usepackage{usrcmd}
\bibliography{bibliography}

% Begin document.
% Use \logoPNG or \logoEPS. If compiling with PDFTeX, use \logoPNG
\begin{document}
% \logoPNG

{\Huge April 24}

\section{Trying to reproduce anti-informative results withtou sinkhorn}

The results obtained with using anti-informative regularization found in
slurm-run/fig/neat/0205_reg-w-infogan/epsW-5/run-1 are the cherry one and can't
be reproduced.

Possible reason (change in the code since) is the adamw vs adam optimizer
relu vs leakyrelu
no bias vs biasD spec

Tests are in progress to reproduce the results, taking the baseline
configuration

If the parameters are not found back, the code will have to be analysed
(checkouted out from the master branch at the time 0205, to the new branch 0205
in git),

The solver behavior should be considered first, and the grad mode (update the FE
network in the network W/D as well) should be envisaged

\section{New publications}

five papers have been printed from the AISTATS conference
http://proceedings.mlr.press/v89/

Orthogonal estimation of wasserstein distances rowland19
wasserstein regularization for sparse mutli-task regression janati19
statistical optimal transport via factored couplings forrow19
sample complexity of sinkhorn divergences genevay19
optimal transport for multi-source domain adaptation under target shift redko19

They are sometimes quite general and should have been read by April 22nd

\section{Jupyter notebook}

One way to keep a diary of the computations would be to keep a jupyter notebook
ofor every day ( that would come and mix with this diary for the text?)

must be investigated more and a solution should have been found by April \nth{22} (\nth{29}).

\end{document}
