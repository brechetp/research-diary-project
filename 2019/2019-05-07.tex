%%% Research Diary - Entry
\documentclass[11pt,a4paper]{article}

% Working date: The date this entry is describing. Not necessary the date of last edit.
\newcommand{\workingDate}{\textsc{2019 $|$ May $|$ 07}}

% Name and institution must preceed call to researchdiary.sty package
\newcommand{\userName}{Pierre Bréchet}
\newcommand{\institution}{TUM}

% The guts and glory
\usepackage{research_diary}
\usepackage{usrcmd}
\bibliography{bibliography}

% Begin document.
% Use \logoPNG or \logoEPS. If compiling with PDFTeX, use \logoPNG
\begin{document}
% \logoPNG

{\Huge May  7}

\section*{Grad propagation}

The arbitration between spec grad mode and FE grad modes was not finished (cf.
notes of 2019-04-23/24). Tests were then not made on Gaussians but on MNIST. The difficulty of
the model to learn the Gaussian dataset with the new formulation of the
informative regularization might comes from a problem at that level.

Checked out working code on Gaussian that produced thesis plots
(fig/thesis-b/good-good-25gaussian-zt-128-sinkhorn-entropy) to new branch 0405 is under
investigation.

\printbibliography{}
\end{document}
