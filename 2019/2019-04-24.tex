%%% Research Diary - Entry
\documentclass[11pt,a4paper]{article}

% Working date: The date this entry is describing. Not necessary the date of last edit.
\newcommand{\workingDate}{\textsc{2019 $|$ April $|$ 24}}

% Name and institution must preceed call to researchdiary.sty package
\newcommand{\userName}{Pierre Bréchet}
\newcommand{\institution}{TUM}

% The guts and glory
\usepackage{research_diary}
\usepackage{usrcmd}
\bibliography{bibliography}

% Begin document.
% Use \logoPNG or \logoEPS. If compiling with PDFTeX, use \logoPNG
\begin{document}
% \logoPNG

{\Huge April 24}

\section{Grad (mode) propagation}
    there is no clear modification when using the FE vs spec gradient mode propagation in the update dual routine

\section{Batch norm for the specialized layers}
    the specialized layers seem to prefer batch normalization
    % \input{2019/04/compare_bn}

\section
    Using a feature extraction mechanism is equivalent to considering the dual networks as factored.

    If the cost is thought to be taken in a feature space $W$, $c(F(x),
    F(y))$ with $F$ the feature extraction network
    The primal problem becomes
    $\min_{\gamma}\int_{\XX}c(F(x), F(y)) \d \gamma (x,y) + \eps \R(\gamma)$, so between $\pf{F}{\mu}$ and $\pf{F}{\nu}$

    The dual solutions are similarly expressed except precomposed with $F$.
    \begin{equation}
        \max_{u, v} \int_{X} u \circ F (x) \d \mu(x) + \int_{X} v \circ F(y) \d \nu(y) - \eps \int_{\ZX}{\exp{\paren{ \frac{1}{\eps} ( u \circ F (x) + v \circ F (y)  - c( u \circ F(x), v \circ F(y)))}} \d \muxnu}
    \end{equation}

    For a fixed $F$ the previous equation gives the dual solutions

\begin{equation}
    D_1 = u \circ F, D_2 = v \circ F
    \argmax_{D_1,D_2}
\end{equation}

\end{document}
